\documentclass[12pt, a4paper, titlepage]{article}

\usepackage[portuguese]{babel}
\usepackage{caption}
\usepackage{float}
\usepackage[a4paper, margin=2cm]{geometry}
\usepackage{graphicx}
\usepackage{hyperref}
\usepackage{setspace}

\chardef\_=`_

\title{\textbf{
    Sistemas Distribuídos -- Trabalho Prático  \\
    \large Base de dados concorrente com acesso remoto
}}
\author{
    \begin{tabular}{ll}
        Carolina Sofia Lopes Queirós Pereira  & A100836 \\
        Diogo Luís Barros Costa               & A100751 \\
        Humberto Gil Azevedo Sampaio Gomes    & A104348 \\
        Sara Azevedo Lopes                    & A104179
    \end{tabular}
}
\date{28 de dezembro de 2024}

\captionsetup{font=onehalfspacing}

\begin{document}

\onehalfspacing
\setlength{\parskip}{\baselineskip}
\setlength{\parindent}{0pt}
\def\arraystretch{1.5}

\maketitle

\begin{abstract}
    O trabalho prático proposto consiste no desenvolvimento de uma base de dados que armazena, em
    memória, associações chave-valor. A base de dados é gerida por um servidor, que serve pedidos de
    clientes comunicados via TCP/IP, garantido a atomicidade das suas execuções. Desenvolveu-se uma
    solução concorrente, utilizando primitivas de sincronização baseadas em monitores. Procurou-se
    diminuir a contenção e minimizar o número de \emph{threads} acordadas, pelo que se implementaram
    diversas estratégias de controlo de concorrência, procurando-se aumentar o desempenho do
    \emph{software} sem sacrificar a sua correção. O desempenho destas várias estratégias foi
    medido com recurso a uma ferramenta de testes também desenvolvida, e os resultados dos
    \emph{benchmarks} realizados apresentam-se neste relatório.
\end{abstract}

\end{document}
