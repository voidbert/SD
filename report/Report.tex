\documentclass[12pt, a4paper, titlepage]{article}

\usepackage[portuguese]{babel}
\usepackage{caption}
\usepackage{float}
\usepackage[a4paper, margin=2cm]{geometry}
\usepackage{graphicx}
\usepackage{hyperref}
\usepackage{setspace}

\chardef\_=`_

\title{\textbf{
    Sistemas Distribuídos -- Trabalho Prático  \\
    \large Base de dados concorrente com acesso remoto
}}
\author{
    \begin{tabular}{ll}
        Carolina Sofia Lopes Queirós Pereira  & A100836 \\
        Diogo Luís Barros Costa               & A100751 \\
        Humberto Gil Azevedo Sampaio Gomes    & A104348 \\
        Sara Azevedo Lopes                    & A104179
    \end{tabular}
}
\date{28 de dezembro de 2024}

\captionsetup{font=onehalfspacing}

\begin{document}

% Use rsvg-convert instead of direct EPS to get font embedding
\immediate\write18{neato -Tsvg dot/Architecture.dot | rsvg-convert -f eps -o dot/Architecture.eps -}

\onehalfspacing
\setlength{\parskip}{\baselineskip}
\setlength{\parindent}{0pt}
\def\arraystretch{1.5}

\maketitle

\begin{abstract}
    O trabalho prático proposto consiste no desenvolvimento de uma base de dados que armazena, em
    memória, associações chave-valor. A base de dados é gerida por um servidor, que serve pedidos de
    clientes comunicados via TCP/IP, garantido a atomicidade das suas execuções. Desenvolveu-se uma
    solução concorrente, utilizando primitivas de sincronização baseadas em monitores. Procurou-se
    diminuir a contenção e minimizar o número de \emph{threads} acordadas, pelo que se implementaram
    diversas estratégias de controlo de concorrência, procurando-se aumentar o desempenho do
    \emph{software} sem sacrificar a sua correção. O desempenho destas várias estratégias foi
    medido com recurso a uma ferramenta de testes também desenvolvida, e os resultados dos
    \emph{benchmarks} realizados apresentam-se neste relatório.
\end{abstract}

\section{Arquitetura do programa}

O \emph{software} desenvolvido foi dividido em componentes, que se apresentam na figura abaixo,
juntamente com as dependências entre si:

\begin{figure}[H]
    \centering
    \includegraphics[width=0.6\textwidth]{dot/Architecture.eps}
    \caption{Dependências entre os vários componentes da solução desenvolvida.}
    \label{architecture}
\end{figure}

A maioria da funcionalidade do \emph{software} está localizada nas bibliotecas \texttt{common} e
\texttt{libserver}:

\begin{itemize}
    \item \texttt{common}
    \begin{itemize}
        \item Mensagens do protocolo de comunicação entre \texttt{client} e \texttt{server}, e a sua
            serialização e deserialização;

        \item Lógica de comunicação do cliente.
    \end{itemize}

    \item \texttt{libserver}
    \begin{itemize}
        \item Várias implementações da estrutura de armazenamento de dados (\emph{backends}), com
            estratégias de controlo de concorrência distintas.

        \item Lógica de comunicação do servidor.
    \end{itemize}
\end{itemize}

Os programas \texttt{client} e \texttt{server} não passam de \emph{wrappers} à volta das bibliotecas
\texttt{common} e \texttt{libserver}, respetivamente. Por outro lado, no programa \texttt{tester}, é
implementada uma \emph{framework} para \emph{benchmarking} tanto dos \emph{backends} em
\texttt{libserver} como da base de dados completa, incluindo comunicação entre \texttt{client} e
\texttt{server}.

\section{\emph{Backends} implementados}

\subsection{\emph{Backends} baseados num único \emph{lock}}

\subsubsection{\texttt{SimpleHashMapBackend}}

\subsubsection{\texttt{MultiConditionHashMapBackend}}

\subsection{\emph{Backend} com \emph{sharding}}

\subsubsection{\texttt{ShardedHashMapBackend}}

\section{Protocolo de comunicação entre \texttt{client} e \texttt{server}}

\section{Sistema de testes}

\section{Conclusão}

\end{document}
